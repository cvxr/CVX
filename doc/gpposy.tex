\documentclass[12pt]{article}
\usepackage{fullpage,url,xspace,fancyvrb}

\DefineVerbatimEnvironment{code}{Verbatim}{tabsize=0,xleftmargin=0.5in}

\bibliographystyle{alpha}

\newcommand{\gpposy}{\texttt{gpposy}\xspace}
\newcommand{\eg}{{\it e.g.}}
\newcommand{\ie}{{\it i.e.}}
\newcommand{\etc}{{\it etc.}}
\newcommand{\reals}{{\mathbf R}}
\newcommand{\ones}{\mathbf 1}

\begin{document}

\title{\gpposy \\ A Matlab Solver for Geometric Programs \\ in Posynomial Form}

\author{ Kwangmoo Koh\\\texttt{deneb1@stanford.edu}
\and Seungjean Kim\\\texttt{sjkim@stanford.edu}
\and Almir Mutapcic\\\texttt{almirm@stanford.edu}
\and Stephen Boyd\\\texttt{boyd@stanford.edu}}

%\date{}
\maketitle

\section{Introduction}

\gpposy solves an optimization problem of the form
\begin{equation} \label{e-gp-posy}
\begin{array}{ll}
\mbox{minimize} & \sum_{k=1}^{K_0} b_k^{(0)} x_1^{a_{k1}^{(0)}} 
              x_2^{a_{k2}^{(0)}} \cdots x_n^{a_{kn}^{(0)}}\\
\mbox{subject to} & \sum_{k=1}^{K_i} b_k^{(i)} x_1^{a_{k1}^{(i)}} 
                x_2^{a_{k2}^{(i)}} \cdots x_n^{a_{kn}^{(i)}}
                \le 1, \quad i = 1, \ldots, m, \\
              & h_{i} x_1^{g_{i1}} x_2^{g_{i2}} \cdots x_n^{g_{in}}
                = 1, \quad i = 1, \ldots, p, \\
              & l \preceq x \preceq u,
\end{array}
\end{equation}
where the optimization variable is the vector 
$x = (x_1, \ldots, x_n) \in \reals_+^n$. The problem data are 
$a_{kj}^{(i)}, g_{ij} \in \reals$, $b_k^{(i)}, h_{i} \in \reals_+$, 
and $l, u \in \reals_+^{n}$. This problem is called a 
\emph{geometric program in posynomial form}.
For more information about geometric programming, see \cite{BV:04, BKVH:05}.

\section{Calling sequences}

The complete calling sequence of \gpposy is
\begin{code}
>> [x,status,lambda,nu] = gpposy(A,b,szs,G,h,l,u,quiet);
\end{code}
Input arguments give the problem data.
Output arguments are the optimal point (if feasible), sensitivity 
information (if feasible) and the solution status.
If the problem is infeasible, the optimal point and sensitivity
are those obtained from the phase~I problem.

\subsection{Input arguments}

\begin{itemize}
\item{\tt A}: matrix with $n$ columns and $\sum_{i=1}^{m} K_i$ rows 
that specifies the exponents of the objective and equality constraints, \ie,
\[
A = \left[\begin{array}{c} A^{(0)} \\ \vdots \\ A^{(m)} \end{array}\right],
\quad
A^{(i)}=
\left[\begin{array}{ccc}
    a_{1 1}^{(i)}   & \cdots & a_{1 n}^{(i)}  \\
    \vdots          & \ddots & \vdots         \\
    a_{K_i 1}^{(i)} & \cdots & a_{K_i n}^{(i)}
\end{array}\right]
\in \reals^{K_i \times n}, \quad i = 0,\ldots,m.
\]
{\tt A} can be in sparse format.

\item{\tt b}: vector of length $\sum_{i=1}^{m} K_i$ that specifies
the coefficients of the objective and inequality constraints, \ie,
\[
b = 
\left[\begin{array}{c} {b^{(0)}} \\ \vdots \\ {b^{(m)}} \end{array} \right],
\quad
b^{(i)}=
\left[\begin{array}{c}
    b_{1}^{(i)} \\ \vdots \\ b_{K_i}^{(i)}
\end{array}\right]
\in \reals_+^{K_i}, \quad i = 0,\ldots,m.
\]
All elements $b_k^{(i)}$ must be positive.

\item{\tt szs}: vector of length $m+1$ that specifies the number of terms
in each objective and inequality constraints, \ie, $(K_0, \ldots, K_m)$.

\item{\tt G}: matrix with $n$ columns and $p$ rows, that specifies
the exponents of the equality constraints, \ie,
\[
G = 
\left[\begin{array}{ccc}
    g_{1 1} & \cdots & g_{1 n} \\
    \vdots  & \ddots & \vdots  \\
    g_{p 1} & \cdots & g_{p n}
\end{array}\right]
\in \reals^{p \times n}.
\]
{\tt G} can be in sparse format.
\item{\tt h}: $p$-vector that contains the coefficients of equality 
constraints, \ie,
\[
h = 
\left[\begin{array}{c} h_{1} \\ \vdots \\ h_{p} \end{array}\right]
\in \reals_+^{p}.
\]
All elements $h_k$ must be positive.

\item{\tt l}: $n$-vector that specifies lower bounds on $x$.
If not given, it will be set to the default bounds
$(10^{-100}, \ldots, 10^{-100})$. All elements $l_i$ must be positive.

\item{\tt u}: $n$-vector that specifies upper bounds on $x$.
If not given, it will be set to the default bounds
$(10^{100}, \ldots, 10^{100})$. All elements  $u_i$ must be positive.

\item{\tt quiet}: boolean. Suppresses print messages during an execution
if \verb+true+. The default value is \verb+false+.

\end{itemize}


\subsection{Output arguments}

\begin{itemize}
\item{\tt x}: $n$-vector. \verb+x+ is the optimal point of the problem
if the problem is feasible, or the last primal iterate
of phase~I if the problem is infeasible.

\item{\tt status}: string; possible values are
\verb+'Solved'+, \verb+'Infeasible'+ and \verb+'Failed'+.

\item{\tt lambda}: vector of length $m+2n$; the 
\emph{optimal sensitivity} or \emph{dual variable}
(see \cite[\S3.3]{BKVH:05}) associated with inequality 
constraints if the problem is 
feasible. The first $m$ elements, {\tt lambda(1:m)}, are optimal 
sensitivities of the $m$ inequality constraints, the next $n$ 
elements, {\tt lambda(m+1:m+n)}, are those of the lower bound 
constraints ($l \preceq x$), and the last $n$ elements, 
{\tt lambda(m+n:m+2*n)}, are those of the upper bound constraints 
($x \preceq u$). If the problem is infeasible, \verb+lambda+ is a 
certificate of infeasibility (see \cite[\S5.8.1,\S11.4.3]{BKVH:05}).

\item{\tt nu}: $p$-vector; the optimal sensitivity vector 
(see \cite[\S3.3]{BKVH:05}) associated with equality constraints 
($Gx + h = 0$) if the problem is feasible. If infeasible, \verb+nu+ 
is a certificate of infeasibility (see \cite[\S5.8.1,\S11.4.3]{BKVH:05}).
\end{itemize}

\subsection{Other calling sequences}

Other calling sequences supported by \gpposy are:
\begin{code}
>> [x,status,lambda,nu] = gpposy(A,b,szs);
>> [x,status,lambda,nu] = gpposy(A,b,szs,G,h);
>> [x,status,lambda,nu] = gpposy(A,b,szs,G,h,l,u);
>> [x,status,lambda,nu] = gpposy(A,b,szs,G,h,l,u,quiet);
>> [x,status,lambda,nu] = gpposy(A,b,szs,[],[],l,u);
>> [x,status,lambda,nu] = gpposy(A,b,szs,[],[],l,u,quiet);
>> [x,status,lambda,nu] = gpposy(A,b,szs,[],[],[],[],quiet);
\end{code}

\subsection{Caveats}

\begin{itemize}
\item The equality constraints matrix, \verb+G+, must be full rank.

\item To solve a large sparse problem, be sure that 
\verb+A+ and \verb+G+ are given in sparse format.

\item The equality constraitns must be explicitly specified as 
$Gx + h = 0$.  You cannot represent equality constraints as pairs of
oppsoing inequality constraints.

\end{itemize}

\section{Example}

Consider the problem
\[ \label{e-gp-posy-example}
\begin{array}{ll}
\mbox{minimize} & x_1^{-1} x_2^{-1/2} x_3^{-1} + 2.3 x_1 x_3 + 4 x_1 x_2 x_3 \\
\mbox{subject to}
& (1/3)x_1^{-2}x_2^{-2} + (4/3) x_2^{1/2} x_3^{-1} \leq 1,\\
& 0.1x_1+0.2x_2+0.3x_3 \leq 1,\\
& (1/2)x_1 x_2 = 1,
\end{array}
\]
with variables $x_1$, $x_2$ and $x_3$.
This problem has the form (\ref{e-gp-posy}) with
\[
A^{(0)} = \left[\begin{array}{ccc}
-1 &  -0.5 &  -1 \\
1 &     0 &   1 \\
1 &     1 &   1
\end{array}\right],
\quad
A^{(1)} = \left[\begin{array}{ccc}
-2 &    -2 &   0 \\
0 &   0.5 &  -1
\end{array}\right],
\quad
A^{(2)} = \left[\begin{array}{ccc}
1 &     0 &   0 \\
0 &     1 &   0 \\
0 &     0 &   1
\end{array}\right],
\]
\[
b^{(0)} = \left[\begin{array}{c}
1 \\ 2.3 \\ 4
\end{array}\right],
\quad
b^{(1)} = \left[\begin{array}{c}
1/3 \\ 4/3
\end{array}\right],
\quad
b^{(2)} = \left[\begin{array}{c}
0.1 \\ 0.2 \\ 0.3
\end{array}\right],
\]
\[
G = \left[\begin{array}{ccc}
1 &  1 &  0
\end{array}\right],
\quad
h = 0.5 .
\]
The Matlab code for solving this problem is as follows:
\begin{code}
% Matlab script that solves the above problem

>> A0  = [  -1 -0.5   -1 ;...
         1    0    1 ;...
         1    1    1 ];
>> A1  = [  -2   -2    0 ;...
         0  0.5   -1 ];
>> A2  = [   1    0    0 ;...
         0    1    0 ;...
         0    0    1 ];
>> A   = [  A0;  A1;  A2 ];

>> b0  = [   1; 2.3;   4 ];
>> b1  = [ 1/3; 4/3 ];
>> b2  = [ 0.1; 0.2; 0.3 ];
>> b   = [  b0;  b1;  b2 ];

>> G   = [   1    1    0 ];
>> h   = [ 0.5 ];
>> szs = [ size(A0,1); size(A1,1); size(A2,1) ]; %i.e., [ 3; 2; 3 ]

>> [x,nu,mu,lambda,status] = gpposy(A,b,szs,G,h);
\end{code}
After executing the code, you can see the result by typing \verb+x+ in Matlab.
\begin{code}
>> x

ans =

3.4783
0.5750
1.1030
\end{code}

\bibliography{gpposy}

\end{document}
